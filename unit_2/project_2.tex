\documentclass[12pt]{article}
\usepackage[margin=1in]{geometry}
\usepackage{url}
\setlength{\parindent}{0em}
\setlength{\parskip}{0.75em}
\usepackage{fancyhdr}
\pagestyle{fancy}
\pagenumbering{gobble}
\rhead{21 March 2016}
\chead{ASTR 670 --- Project 2 (paper summary)}
\lhead{Laurel Farris}

%\usepackage[super]{natbib}
%\usepackage[superscript,biblabel]{cite}
\usepackage{natbib}
\bibliographystyle{plain}
\usepackage{hyperref}

\begin{document}
\begin{centering}
    {\large ``THE CONFINED X-CLASS FLARES OF SOLAR ACTIVE REGION 2192''}
\end{centering}
\begin{centering}
    -J. K. Thalmann, Y. Su, M. Temmer, and A. M. Veronig
\end{centering}

% Question being studied in the paper
In October of 2014, active region NOAA 2192 on the surface of the sun
was accompanied by an intriguing set of solar flares 
(\cite{main_paper}).
This particular active region (AR) was quite exceptional;
it was the largest seen in almost 25 years,
since the appearance of NOAA 6368 in November of 1990.
It also emerged during a time of solar \emph{minimum}, when magnetic
activity on the sun tends to be at its lowest, and very few
ARs (such as sunspots and prominences) are present on the surface.
Most puzzling of all, these flares showed an abnormal lack of
violent ejections of material that usually accompany the strongest flares
(a process known as a \emph{coronal mass ejection}, or CME).
Flares and CMEs both result from the rapid release of
magnetic energy, and while it is not impossible for the two
to occur independently, the likelihood of them occurring
simultaneously increases as their energy increases, up to
90\% for flares of class $\geq$ X1.
These deviations from the usual solar activity
served as the motivation for the investigation of the flares of
AR NOAA 2192,
which included
a single, eruptive M4.0 flare that was accompanied by a CME,
and several confined X-class flares that were not, despite being
stronger.

% Proposed data and method of study
AR 2192 traveled across the visible region of the sun from
October 17--30. The period from October 22--24
was used in the current analysis, with data in a variety of
wavelength regimes, including
EUV images from the \emph{Atmospheric Imaging Assembly (AIA)},
magnetic data from the \emph{Helioseismic and Magnetic Imager (HMI)}
(both of which are instruments on board the
\emph{Solar Dynamics Observatory (SDO)}),
ground-based H$\alpha$ filtergrams from the Kanzelh\"{o}he
Observatory (KSO),
X-ray spectra from the
\emph{Reuven Ramaty High-Energy Solar Spectroscopic Imager (RHESSI)}
and soft X-rays from the \emph{GOES} satellite.
1700 \AA{} and 1600 \AA{} images from \emph{AIA}
were both used primarily for locating and analyzing the flare ``ribbons''
(a term used to describe the visual features that usually precede
the onset of a flare). The 94 \AA{} filters were used to sample
the ``hot coronal flare plasma'' at greater coronal heights.
The data from \emph{HMI} was used to monitor magnetic field strengths.  
X-ray spectra from \emph{RHESSI} revealed steep power-law distributions for
electron energies throughout the duration of the flare, while the
soft X-ray data showed a duration longer than normal for the X-class
flares.

% Main result
%The confined X-class flares lasted for an uncharacteristically long
%duration.
Four main conclusions were drawn as a result of this study.
The ejection of mass that was expected to occur from high
class flares was probably prohibited by a magnetic ``arcade'',
or a set of \emph{closed} magnetic field loops over the X-class
flares in the AR.
The M4.0 flare happened to be located closer to open field lines.
The initially high separation distance between flare ribbons
($\approx$ 50 Mm, compared to the usual separation on the order of
just a few Mm), was an indication 
of magnetic reconnection taking place high in the corona.
Many electrons were accelerated to non-thermal energies at the
reconnection site,
but only a small fraction were accelerated to high energies.
Additionally, they found that the same magnetic structures were
responsible for many reconnection occurances, as indicated by
the flare pixels undergoing multiple ``re-brightenings,'' or
re-energizations, throughout the observed lifetime of the AR\@.

\newpage
\bibliography{reffile}
\end{document}
