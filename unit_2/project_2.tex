\documentclass[12pt]{article}
\usepackage[margin=1in]{geometry}
\setlength{\parindent}{0em}
\setlength{\parskip}{1em}
\usepackage{fancyhdr}
\pagestyle{fancy}
\pagenumbering{gobble}
\rhead{21 March 2016}
\chead{ASTR 670 --- Project 2 (paper summary)}
\lhead{Laurel Farris}

%\usepackage[super]{natbib}
%\usepackage[superscript,biblabel]{cite}
\usepackage{natbib}
\bibliographystyle{plain}
\usepackage{hyperref}

\begin{document}
\begin{centering}
    {\large ``THE CONFINED X-CLASS FLARES OF SOLAR ACTIVE REGION 2192''}
\end{centering}
\begin{centering}
    -J. K. Thalmann, Y. Su, M. Temmer, and A. M. Veronig
\end{centering}

The subject of this paper was a particularly intriguing
set of solar flares
that accompanied active region (AR) NOAA 2192 in October of 2014.
This particular AR was exceptional for several reasons.
It was the largest seen in about 25 years,
since the appearance of NOAA 6368 in November of 1990.
It appeared during a time of solar \emph{minimum}, when magentic
activity on the sun tends to be at its lowest, and very few
ARs, such as sunspots and flares, are seen on the surface.
Perhaps most puzzling is,
while these flares accelerated electrons to energies higher than normally
observed for the most energetic flares (namely, those of class X),
they did not cause the violent ejection of material,
a process known as a \emph{coronal mass ejection}, or CME\@.
While it is not impossible for flares and CMEs
to occur independently, it is far more likely for them to occur
simultaneously as the energies of the two events increases.
In this study, a flare of class M4.0 was accompanied by a CME
(labelled as an ``eruptive'' flare),
yet some of the stronger, X-class flares (labelled as
``confined'' flares) were not.
These deviations from the usual solar activity
served as the motivation for the study.

The data used for this project covered the period from October 22
through October 24.
During this time, a series of flares of class $\geq$ M5 were produced,
including the single, eruptive M4.0 flare and several
confined X-class flares.
The data set included a wide variety of wavelength regimes, including
X-ray images from the
\emph{Reuven Ramaty High-Energy Solar Spectroscopic Imager (RHESSI)},
EUV images from the
\emph{Atmospheric Imaging Assembly (AIA)} and
magnetic data from the
\emph{Helioseismic and Magnetic Imager (HMI)}, (both on board the
\emph{Solar Dynamics Observatory (SDO)}), and some ground-based
filtergrams in H$\alpha$ from the Kanzelh\"{o}he Observatory (KSO).
Each AIA narrow-bandpass samples the corona at different temperatures
(and thus different heights), with the exception of the 1700 \AA{}
and 1600 \AA{} wavelengths, which are continuum filters that
sample the photosphere and transition region (TR) and are used for
this project mainly for data reduction purposes.
94 \AA{} filters to sample the ``hot coronal flare plasma''
at greater coronal heights.

Four main conclusions were drawn at the close of this study.
The ejection of mass that was expected to occur from such high
class flares was possibly prohibited by
a magnetic ``arcade'', or a set of \emph{closed} magentic field loops over
the AR\@. The M4.0 flare happened to be located closer to open field lines.
%000% High flare separation --> reconnection site high in corona
The high separation distance between flare ribbons indicated a site
of magnetic reconnection high in the corona.
%000% Energies
While many electrons were accelerated to non-thermal energies, only
a small fraction were accelerated to energies high enough for
something.
Additionally, they found that the same magenetic structures were
responsible for many reconnection occurances, as indicated by
the flare pixels undergoing multile ``re-brightenings'' throughout
the observed lifetime of the AR\@.




%\bibliography{reffile}
\end{document}
