\documentclass[12pt]{article}
\usepackage[margin=1in]{geometry}
\usepackage{amsmath}
\usepackage{mdwlist}
\usepackage{fancyhdr}
\pagestyle{fancy}
\pagenumbering{gobble}
\rhead{21 March 2016}
\chead{ASTR 670 --- Project 2 (cume question)}
\lhead{Laurel Farris}

\setlength{\parindent}{0em}
\setlength{\parskip}{0.75em}

\begin{document}
Constants and conversions that may or may not be helpful:
\begin{itemize*}
    \item Boltzmann constant: k = $1.38\times10^{-16}$ erg K$^{-1}$
    \item 1 Joule (J) = $10^7$ erg
\end{itemize*}

\begin{enumerate}
    \item This question probes your knowledge of solar flares.
        \begin{enumerate}
            \item \textbf{3 points:} The authors use a certain notation to
                describe the different classes of flares; for example, in
                \S 1, they mention \emph{GOES} class C1-, M1-, and X1-flares.
                In general, what does this notation mean
                (in other words, what does each of the terms represent),
                and how does each class relate to the others?

            \item \textbf{5 points:} The authors state that the total
                \emph{non}-thermal electron
                energy is $1.6\times10^{25}$ joules. Calculate the
                temperature this energy would correspond to if it was
                \emph{thermal} energy, and explain why this value is not
                reasonable. (Estimate the number density of electrons in the
                corona to be $n_e \sim$ $10^{10}$ cm$^{-3}$, and use a flare
                volume of $V \sim$ $10^{28}$ cm$^3$).

            \item \textbf{2 points:} In \S 4, the authors mention that events with ``unusually
                high electron energy'' have been shown to be efficient particle
                accelerators. What is the significance of this statement,
                as it applies to solar physics in general?

        \end{enumerate}
\end{enumerate}


\newpage
\begin{enumerate}
    \item This question probes your knowledge of solar flares.
        \begin{enumerate}
            \item \textbf{3 points:} The authors use a certain notation to
                describe the different classes of flares; for example, in
                \S 1, they mention \emph{GOES} class C1-, M1-, and X1-flares.
                In general, what does this notation mean
                (in other words, what does each of the terms represent),
                and how does each class relate to the others?

                {\small \underline{Answer}: Solar flares are divided
                into five classes: A, B, C, M, and X. These classes are
                scaled logarithmically; for instance, a class C flare has
                10$\times$ more energy than a class B flare, and
                100$\times$ more energy than a class A flare, etc.\ The
                classes are further subdivided into categories 1--9.

                (\emph{1 point for the classes A, B, C, M, and X\@; 1 point
                for knowing the logarithmic scale; 1 point for knowing
                the subcategories 1--9})}.

            \item \textbf{5 points:} The authors state that the total
                \emph{non}-thermal electron
                energy is $1.6\times10^{25}$ joules. Calculate the
                temperature this energy would correspond to if it was
                \emph{thermal} energy, and explain why this value is not
                reasonable. (Estimate the number density of electrons in the
                corona to be $n_e \sim$ $10^{10}$ cm$^{-3}$, and use a flare
                volume of $V \sim$ $10^{28}$ cm$^3$).

                {\small \underline{Answer}: Using the equation for thermal
                    energy density: $\frac{E}{V} = 3nkT$, and converting energy
                    from joules to ergs, solve for $T$:
                    \begin{align*}
                        T &= \frac{E}{3Vnk}\\
                        &= \frac{10^{32}}{3(10^{28})(10^{10})(1.38\times10^{-16})}\\
                        &= \ \approx 2.415\times10^{9}\ \textrm{K}
                    \end{align*}
                    We do not observe temperatures this high in the corona
                    (the highest is $\sim$ 10$^{6}$ K), so
                    the energy cannot be thermal.

                    (\emph{3 points for knowing the equation for thermal energy,
                        2 points for the explanation}).}

            \item \textbf{2 points:} In \S 4, the authors mention that events with ``unusually
                high electron energy'' have been shown to be efficient particle
                accelerators. What is the significance of this statement,
                as it applies to solar physics in general?

                {\small \underline{Answer}: One of the biggest questions in
                solar physics is what drives the solar wind and
                the heating of the corona. The source of particle acceleration
                has been extensively studied, but is not firmly conclusive.

                (\emph{1 point for addressing the coronal heating problem, and 1
                for the source of the solar wind, though credit for other answers
                is also possible}).}
        \end{enumerate}
\end{enumerate}

\end{document}
