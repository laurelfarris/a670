\documentclass[12pt]{article}
\usepackage[margin=1in]{geometry}
\usepackage{fancyhdr}
\pagestyle{fancy}
\pagenumbering{gobble}
\rhead{10 February 2016}
\chead{ASTR 670 - Project 1 (cume question)}
\lhead{Laurel Farris}

\setlength{\parindent}{0em}
\setlength{\parskip}{0.75em}

\begin{document}

\textbf{
1. (14 points)
In section 2, the authors mention that they carried out their
simulations using a code that solves Maxwell's equations, particularly
Ampere's law. The differential form of the Ampere-Maxwell law is
    $$ \vec{\nabla}\times\vec{B} = \mu_0\Big( \vec{J} + \epsilon_0\frac
    {\partial \vec{E}}{\partial t}\Big) $$
Please, write down what each of the following terms represents:
 $\vec{\nabla}\times\vec{B},\ \mu_0,\ \vec{J},\
 \epsilon_0,\ \frac{\partial\vec{E}}{\partial t}$
(2 points each).
}

\textbf{Briefly explain what the Ampere-Maxwell law is. (2 points)}

\textbf{How does it apply to the work carried out by the authors?
(2 points)}

\newpage
\textbf{
1. (14 points)
In section 2, the authors mention that they carried out their
simulations using a code that solves Maxwell's equations, particularly
Ampere's law. The differential form of the Ampere-Maxwell law is
    $$ \vec{\nabla}\times\vec{B} = \mu_0\Big( \vec{J} + \epsilon_0\frac
    {\partial \vec{E}}{\partial t}\Big) $$
Please, write down what each of the following terms represents:
 $\vec{\nabla}\times\vec{B},\ \mu_0,\ \vec{J},\
 \epsilon_0,\ \frac{\partial\vec{E}}{\partial t}$
(2 points each).
}

\underline{Answer:}
\begin{itemize}
    \item $\vec{\nabla}\times\vec{B}$: curl of the magnetic field
    \item $\mu_0$: magnetic permeability of free space
    \item $\vec{J}$: electric current density
    \item $\epsilon_0$: electric permittivity of free space
    \item $\frac{\partial\vec{E}}{\partial t}$: rate of change of
    electric field with time
\end{itemize}

\textbf{Briefly explain what the Ampere-Maxwell law is. (2 points)}

\underline{Answer:} The Ampere-Maxwell law
relates the spatial variation of a magnetic field,
to the accompanying electric
current, and possibly a changing electric field
(unless this change is negligibly small).
If the latter is negligibly small,
then it goes to zero and the equation reduces to
$ \vec{\nabla}\times\vec{B} = \mu_0\vec{J} $.

\textbf{How does it apply to the work carried out by the authors?
(2 points)}

\underline{Answer:} The authors are investigating the acceleration of
electrons as a result of magnetic recconection.
The $\vec{\nabla}\times\vec{B}$ term in the equation
represents this change in the magnetic field,
and the $\vec{J}$ term represents the induced current, or
flow of charged particles (hence the movement and
acceleration of electrons).

\newpage
Grading scheme

\end{document}
