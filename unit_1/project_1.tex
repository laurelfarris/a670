\documentclass[12pt]{article}
\usepackage[margin=1in]{geometry}
\usepackage{fancyhdr}
\pagestyle{fancy}
\pagenumbering{gobble}
\rhead{10 February 2016}
\chead{ASTR 670 - Project 1 (paper summary)}
\lhead{Laurel Farris}

\setlength{\parindent}{0em}
\setlength{\parskip}{0.75em}

\begin{document}

\begin{center}\large\textbf{
``Nonthermally Dominated Electron Acceleration during Magnetic
Reconnection in a Low-\boldmath$\beta$ Plasma''}\\
\normalsize-Xiaocan Li, Fan Guo, Hui Li, and Gang Li
\end{center}
% The question being studied
The primary question addressed in this work was the
mechanism by which electrons are accelerated to
nonthermal energies during magnetic reconnection in a nonrelativistic
plasma.
The conversion from magnetic energy to plasma kinetic energy
that generally occurs during reconnection is well understood,
but how the electrons are accelerated to the point where their
energies are represented by a power-law, rather than a Maxwell
distribution, is not.
Two possible mechanisms thought to drive these accelerations were
the \emph{Fermi} mechanism, which is known to drive energy gain
in other events, such as shocks, and direct acceleration in the diffusion
region.
These high accelerations are seen on the sun during solar flares,
and investigating the cause can lead to a better understanding of
these types of events.

%The proposed data and method of study
%Kinetic simulations, low $\beta$ regime, power law dist,
This
question was investigated by running kinetic simulations with the
`VPIC' code.
Initial conditions included
equal values of $\beta$ for both electrons and ions (protons),
and Maxwellian speed distributions characteristic of (initially)
thermal particles.
Other variables included the mass ratio of electrons to ions,
electron drift velocity, and several plasma parameters, such as
gyrofrequency, plasma frequency, and electron/ion inertial length.
Magnetic reconnection was induced by adding
a wavelength perturbation.
Previous work focused on $\beta$ values greater than $\sim$0.1
(lower values were unexplored due to lack of
computational abilities).
Here, the authors were able to investigate $\beta$ values of
0.007, 0.02, 0.06, and 0.2. The output of these simulations included
the rate of energy conversion, the change in current density,
and the overall increase in energy for each $\beta$.

%The main result
After running the simulations,
the \emph{Fermi} mechanism was found to be the dominant accelerator,
and that
lower values of
$\beta$ resulted in higher energy gain.
Figure 1(d) provides a good illustration of
the increase in energy for the four different
values of $\beta$. The lowest, 0.007, produced an energy
increase greater than ten times the original energy. 
The authors
concluded with two constraints that were required to produce
a power-law electron energy distribution.
First,
the reconnection process itself required a timescale
long enough for the electrons to aquire the appropriate energy
distribution.
Second, the $\beta$ parameter had to be sufficiently low.
As $\beta$ is expressed by
    $$ \beta = \frac{P_{th}}{P_{mag}} = \frac{nk_BT}{B^2/8\pi}
        \propto \frac{nT}{B^2}$$
its value could be lowered either by decreasing the particle number
density $n$, decreasing the temperature $T$ (which was found to have a
negligible effect),
or increasing the strength of the magnetic field $B$.
Possible future investigations included ion acceleration in
addition to electron acceleration,
the addition of an external guide field in the
simulations (which could result in three-dimensional instabilities),
and
particle loss mechanisms.
\end{document}
