\documentclass[12pt]{article}
\usepackage[margin=1in]{geometry}
\usepackage{fancyhdr}
\pagestyle{fancy}
\pagenumbering{gobble}
\rhead{03 February 2016}
\chead{ASTR 670 - Project 1}
\lhead{Laurel Farris}

\setlength{\parindent}{0em}
\setlength{\parskip}{0.75em}

\begin{document}

\begin{center}\large\textbf{
``Nonthermally Dominated Electron Acceleration during Magnetic
Reconnection in a Low-\boldmath$\beta$ Plasma''}
\end{center}
The primary physical phenomenon addressed by the authors is the
acceleration of particles, specifically the mechanism by which
electrons and protons with nonthermal energies are accelerated
in regions of magnetic reconnection.
Emission from solar flares has indicated that
these energies, are considerably higher than thermal energies, and
it is unclear exactly how they are obtaining such high energies.
Two mechanisms had been previously explored:
the \emph{Fermi} mechanism (need source)
and direct acceleration (another source).
In the past, low-$\beta$ simulations were difficult to produce
(presumably because of a lack of compuational ability)
so the authors investigated $\beta$ values between 0.007 and 0.2.
The particles were modeled in a ``fully relativistic manner'',
whatever that means.

%The proposed data and method of study
%Kinetic simulations, low $\beta$ regime, power law dist,
Hoping to obtain a power-law model to match observations, this
problems was investigated using kinetic simulations in a code 
that solved Maxwell's
equations, namely, Ampere's law. Starting with certain initial
conditions: equal values of $\beta$ for both electrons and protons, a
mass ratio of ions (protons) to electrons of 25, and Maxwellian speed
distributions all around, magnetic reconnection was induced by adding
a wavelength perturbation, and changing $\beta$ as a function of the
electron plasma frequency (source) and the electron gyrofrequency
(source). This was done for four different values of $\beta$,
which were determined somehow.
The results from these simulations included the rate at which energy
was transformed from magnetic energy before reconnection to nonthermal
energy after reconnection. 

%The main result
After plotting all the plots that they plotted, the authors
concluded with two main constraints on the production of ``power-law
electron distribution'' (need to word this better). The first is that
the reconnection process itself requires a timescale of around
something in order for the electrons to aquire the resulting
distribution. Second, the $\beta$ parameter has to be sufficiently
low, qualitatively meaning that the magnetic pressure dominates the
thermal pressure by a factor of something. As $\beta$ is expressed by
    $$ \beta = \frac{P_{th}}{P_{mag}} = \frac{nk_BT}{B/8\pi}
        \propto \frac{n}{B}$$
its value could be lowered either by decreasing the particle number
density ($n$) or increasing the strength of the magnetic field ($B$).
The temperature was also an adjustable parameter, but was not found to
have much of an effect on $\beta$, nor did the size of the system used
during the simulations. This makes sense, as we learned in
class that the low power of T compared to B means that B dominates.
(source: class notes? look this up!). 
As shown in figure 1(d), the lower $\beta$ values result in a greater
energy increase: a difference of more than ten times the original
kinetic energy for the lowest value of 0.007. 
The \emph{Fermi} mechanism was found to be the dominant accelerator.
The low values of $\beta$ produced the power-law energy
distribution, where said energy came from the conversion of the magnetic
energy that was present before reconnection to the nonthermal energy
possessed by the electrons after reconnection.
These findings may help to expain the electron acceleration present in
major events such as solar flares and the magnetosphere of the Earth.

\newpage
In section 2, the authors mention that they carried out their
simulations using a code that solves Maxwell's equations. Using what
you know about Maxwell's equations, which one(s) do you think they
needed to solve, and why?

In section 4, the authors mention that the results may be
influenced by three-dimensional instabilities, such as the kink
instability. What is the kink instability? Why might it have an effect
on the science being done here?

\end{document}
