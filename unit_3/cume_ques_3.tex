\documentclass[12pt]{article}
\usepackage[margin=1in]{geometry}
\usepackage{amsmath}
\usepackage{mdwlist}
\usepackage{fancyhdr}
\pagestyle{fancy}
\pagenumbering{gobble}
\rhead{25 April 2016}
\chead{ASTR 670 --- Project 3 (cume question)}
\lhead{Laurel Farris}

\setlength{\parindent}{0em}
\setlength{\parskip}{0.75em}

\begin{document}
Constants and conversions that may or may not be helpful:
\begin{itemize*}
    \item Boltzmann constant: k = $1.38\times10^{-16}$ erg K$^{-1}$
    \item 1 Joule (J) = $10^{7}$ erg
    \item sound speed: $c_s = \sqrt{\frac{\gamma{P}}{\rho}}$;
        $\gamma$ = $\frac{5}{3}$
    \item ideal gas law: $P = nkT = \frac{\rho{kT}}{m_u{\mu}}$
    \item solar mass: $M_{\odot} = 2\times10^{33}$ g
    \item solar radius: $R_{\odot} = 6.96\times10^{10}$ cm
    \item gravitational constant: $G = 6.67\times10^{-8}$ cm$^{3}$ g$^{-1}$ s$^{-2}$
    \item gravitational force: $\vec{F}_{g} = m\vec{g} = \frac{GMm}{r^{2}}\hat{r}$
    \item pressure scale height: $H = \frac{kt}{m_{u}g}$
\end{itemize*}

\begin{enumerate}
\item This question addresses the concepts and calculations of the scale height.
\begin{enumerate}
    \item \textbf{2 points:} In one or two sentences, explain
        qualitatively what the scale height is.

    \item \textbf{4 points:} \S{} 2.1 gives the pressure scale
        height as $H_{P} = c^{2}/(\gamma{g})$. Show that this is
        equal to $H = kT/m_{u}g$. State any assumptions you make.

    \item \textbf{5 points:} Using the expression given for scale
        height ($H_{P} = c^{2}/(\gamma{P})$) and the values given at
        the bottom of the second column of page 2, calculate the
        scale height at the photosphere (the visible surface of
        the sun). If you don't know $g$, use $10^{4}$.
        Does your answer make sense? Why or why not?

    %\item \textbf{x points:} 
\end{enumerate}
\end{enumerate}

\newpage

\begin{enumerate}
\item \emph{Solutions}
\begin{enumerate}
    \item \textbf{2 points:} In one or two sentences, explain
        qualitatively what the scale height is.

        {\small\underline{Answer}: The scale height is the distance
            over which a quantity decreases by a factor of $1/e$.
            This was in the paper!

        (\emph{1 point for something about a quantity decreasing over
        a distance, 1 point for knowing the dropoff is exponential.})}

    \item \textbf{4 points:} \S{} 2.1 gives the pressure scale
        height as $H_{P} = c^{2}/(\gamma{g})$. Show that this is
        equal to $H = kT/m_{u}g$. State any assumptions you make.

        {\small\underline{Answer}:
            Using the sound speed $c_s = \sqrt{\frac{\gamma{P}}{p}}$,
            and the ideal gas law $P = \frac{\rho{kT}}{m_{u}}$
            (assuming the chemical abundance, $\mu$, is equal to 1 in
            the photosphere):
            \begin{align*}
             H_{P} &= \frac{c_s^2}{\gamma{g}}\\
                &= \frac{(\gamma{P}/\rho)}{\gamma{g}}\\
                &= \frac{P}{\rho{g}}\\
                &= \frac{\rho{kT}}{m_{u}\rho{g}}\\
                &= \frac{kT}{m_{u}{g}}\\
            \end{align*}
        (\emph{2 points for utilizing the correct equations
        (even if conversion errors led to wrong expression), 1 more for deriving
        correct expression, 1 for assuming $\mu=1$.})}

    \item \textbf{5 points:} Using the expression given for scale
        height ($H_{P} = c^{2}/(\gamma{P})$) and the values given at
        the bottom of the second column of page 2, calculate the
        scale height at the photosphere (the visible surface of
        the sun). If you don't know $g$, use $10^{4}$.
        Does your answer make sense? Why or why not?

        {\small\underline{Answer}: 
            $g$ can be calculated from the equation for gravitational
            force:
            \begin{align*}
                mg &= \frac{GMm}{r^{2}}\\
                 g &= \frac{GM}{r^{2}}\\
                   &= \frac{GM_{\odot}}{R_{\odot}^{2}}\\
                   &= \frac{(6.96\times10^{-8})(2\times10^{33})}
                       {{(6.96\times10)}^{2}}\\
                   &= 27442.78\ \textrm{cm\ s}^{-2}\\
            \end{align*}
            $c_{s}$ is given as 7.2 km s$^{-1}$ = $7.2\times10^{5}$ cm s$^{-1}$.
            Now solve for $H_{P}$:
            \begin{align*}
                H_{P} &= \frac{c^{2}}{\gamma{g}}\\
                    &= \frac{(7.2)^2}{(5/3)27442.78}\\
                   &=\ \sim 113.34\times10^{5}\ \textrm{cm}\\ 
                   &=\ \sim 113\ \textrm{km}\\
            \end{align*}
            The sharp discontinuity at the photosphere indicates that the
            density, and therefore pressure, falls off quite rapidly, so
            one would expect a `relatively' low scale height.
            (Compare this to the radius of the Earth, at $\sim$ 6300 km).

        (\emph{2 points for correctly using the gravitational force to
        get $g$, 2 points for calculating $H_{P}$ correctly,
        2 points for explaining
        why the answer does or does not make sense.})}

    %\item \textbf{x points:} 
\end{enumerate}
\end{enumerate}

%\begin{enumerate}
%    \item This question probes your knowledge of wave analysis
%\end{enumerate}
\end{document}
