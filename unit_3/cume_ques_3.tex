\documentclass[12pt]{article}
\usepackage[margin=1in]{geometry}
\usepackage{amsmath}
\usepackage{mdwlist}
\usepackage{fancyhdr}
\pagestyle{fancy}
\pagenumbering{gobble}
\rhead{?? April 2016}
\chead{ASTR 670 --- Project 3 (cume question)}
\lhead{Laurel Farris}

\setlength{\parindent}{0em}
\setlength{\parskip}{0.75em}

\begin{document}
Constants and conversions that may or may not be helpful:
\begin{itemize*}
    \item Boltzmann constant: k = $1.38\times10^{-16}$ erg K$^{-1}$
    \item 1 Joule (J) = $10^{7}$ erg
    \item sound speed: $c_s = \sqrt{\frac{\gamma{P}}{\rho}}$;
        $\gamma$ = $\frac{5}{3}$
    \item ideal gas law: $P = nkT = \frac{\rho{kT}}{m_u{\mu}}$
\end{itemize*}

% Solutions and Answers. Copy to first page and delete answers as soon
% as you're finished.

\begin{enumerate}
    %\item \textbf{x points total}
    %\item This question $\ldots$
        \begin{enumerate}
            \item \textbf{2 points:} In one or two sentences, explain
                qualitatively what the scale height is.

                {\small\underline{Answer}: The scale height is the distance
                    over which a quantity decreases by a factor of $1/e$.

                (\emph{1 point for something about a quantity changing over
                a distance, 1 point for knowing the dropoff is exponential.})}

            \item \textbf{3 points:} \S{} 2.1 gives the pressure scale
                height as $H_{P} = c^2/(\gamma{g})$. Show that this is
                equal to $H = kT/m_{u}g$.

                {\small\underline{Answer}:
                    Using the sound speed $c_s = \sqrt{\frac{\gamma{P}}{p}}$,
                    \begin{align*}
                        H &= \frac{c_s^2}{\gamma{g}} \\
                          &= \frac{(\gamma{P}/\rho)}{\gamma{g}} \\
                         &= \frac{P}{\rho{g}} \\
                         &= \frac{\rho{kT}}{m_u\rho{g}} \\
                         &= \frac{kT}{m_u{g}}
                    \end{align*}

                (\emph{2 points for using correct equations (if math errors
                led to wrong expression), 1 more for deriving
                correct expression.})}

        \end{enumerate}
\end{enumerate}


%\begin{enumerate}
%    \item This question probes your knowledge of wave analysis
%\end{enumerate}
\end{document}
