\documentclass[12pt]{article}
\usepackage[margin=1in]{geometry}
\setlength{\parindent}{0em}
\setlength{\parskip}{0.75em}
\usepackage{fancyhdr}
\pagestyle{fancy}
\pagenumbering{gobble}
\rhead{25 April 2016}
\chead{ASTR 670 --- Project 3 (paper summary)}
\lhead{Laurel Farris}

%\usepackage[super]{natbib}
%\usepackage[superscript,biblabel]{cite}
\usepackage{natbib}
\bibliographystyle{plain}
\usepackage{hyperref}

\begin{document}
\begin{center}
    ``ACOUSTIC WAVES GENERATED BY IMPULSIVE DISTURBANCES IN A
    GRAVITATIONALLY STRATIFIED MEDIUM''\\
    -Jongchul Chae \& Philip R. Goode
\end{center}


% Main topic
There are two well-known oscillations detected from the sun: the five-minute
oscillation period attributed to global pressure modes (or ``p-modes'')
in the solar interior, and the
lesser-known three-minute oscillations in the chromosphere.
Like the interior p-modes, the chromospheric oscillations are thought to
be standing oscillations, waves that are trapped between two nodes from which
they reflect.
In this case, the photosphere and the transition region (TR) between the
chromosphere and corona would be such reflecting nodes. However, signatures
of these oscillations have been detected in the low corona, indicating that
they are able to propagate through the TR\@.
In this study, the authors
investigate an alternative explanation to the idea that the cutoff frequency
at the temperature minimum can explain the three-minutes oscillations, since
it doesn't explain why
frequencies just above the cutoff are largely evanescent.
They suggest that the medium through which the waves are
travelling manifests its response to some type of excitation
%(such as a flare or coronal mass ejection (CME))
in the form of the waves that are then observed. This excitation would have
to be a continuous supply of energy and have the result that
$k \ll \omega_{0}/C_{s}$ (where $k$ is the wavenumber, $\omega_0$ is the cutoff
oscillation frequency, and $C_s$ is the sound speed), which gives frequencies around
the cutoff.

% Data/Methods
To investigate the response of a medium to excitations, a simple,
initially static model atmosphere was created.
Dimensionless parameters for wave velocity,
mass density, pressure, and the displacement due to perturbation
of the medium were evolved through time after a perturbation was
introduced.
Important relations such as the wave equation and
dispersion relation, along with reasonable boundary and initial
conditions produced a solution for the wave speed at any time and
any height in the atmosphere. Using conservation of energy
(i.e.\ the total energy in the system remained constant),
two limiting cases were produced: one with most of the energy
going to components with small wavenumbers or large wavenumbers,
were the small wavenumber components were expected to be the source
of the three-minute oscillations.
The simulations addressed the dispersive behavior of the waves in
a gravitationally stratified medium ($g$ changes with height),
the energy carried and dispersed by the waves, and 
the possible sources of this
energy that could result in continuous oscillations.


% Results
The three-minute periods were found to be freely propagating
(i.e.\ no driving force present after the initial excitation)
dispersive waves whose frequency rapidly dropped due to energy loss.
The damping of high-frequency waves first, followed by
lower frequency waves later,
explained the drop in power beyond the cutoff frequency,
where the transport of mechanical energy became quite inefficient.
The low-frequency waves moved with a group velocity that was so low
($v_{gr} \ll c_{s}$), they appeared to `linger' at a given location,
giving the false impression of being standing oscillations.
The source of energy powering these
oscillations was attributed to either the continuous
photospheric motions caused by convective turbulence
in the solar interior, or velocity impulses from impulsive disturbances
in the upper photosphere.

%\bibliography{reffile}
\end{document}
