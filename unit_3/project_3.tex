\documentclass[12pt]{article}
\usepackage[margin=1in]{geometry}
\setlength{\parindent}{0em}
\setlength{\parskip}{0.75em}
\usepackage{fancyhdr}
\pagestyle{fancy}
\pagenumbering{gobble}
\rhead{25 April 2016}
\chead{ASTR 670 --- Project 3 (paper summary)}
\lhead{Laurel Farris}

%\usepackage[super]{natbib}
%\usepackage[superscript,biblabel]{cite}
\usepackage{natbib}
\bibliographystyle{plain}
\usepackage{hyperref}

\begin{document}
\begin{center}
    ``ACOUSTIC WAVES GENERATED BY IMPULSIVE DISTURBANCES IN A
    GRAVITATIONALLY STRATIFIED MEDIUM''\\
    -Jongchul Chae \& Philip R. Goode
\end{center}


% Main topic
There are two well-known oscillations detected from the sun: the five-minute
oscillation period attributed to global pressure modes (or ``p-modes'')
in the solar interior, and the
lesser-known three-minute oscillations in the chromosphere.
Like the interior p-modes, the chromospheric oscillations are thought to
be standing oscillations, waves that are trapped between two nodes from which
they reflect.
In this case, the photosphere and the transition region (TR) between the
chromosphere and corona would be such reflecting nodes. However, signatures
of these oscillations have been detected in the low corona, indicating that
they are able to propagate through the TR\@.
In this study, the authors
investigate an alternative explanation to the idea that the cutoff frequency
at the temperature minimum can explain the three-minutes oscillations, since
it doesn't explain why
frequencies just above the cutoff are largly evanescent.
They suggest that the medium through which the waves are
travelling manifests its response to some type of excitation
%(such as a flare or coronal mass ejection (CME))
in the form of the waves that are then observed. This excitation would have
to be a continuous supply of energy and have the result that
$k \ll \omega_0/C_s$ (where $k$ is the wavenumber, $\omega_0$ is the cutoff
oscillation frequency, and $C_s$ is the sound speed), which gives frequencies around
the cutoff.

% Data/Methods
An initially static theoretical model of a solar atmosphere was
modeled using the assumptions of stable gravitational stratification
and an isothermal environment.
Dimensionless parameters for wave velocity,
mass density, pressure, and the displacement due to perturbation
of the medium were evolved in time after a perturbation was
introduced.
Important relations such as the wave equation and
dispersion relation, along with reasonable boundary and initial
conditions, a solution was derived for the wave speed.

% Results
The outcome of this simulation was a result of the two limiting
parameters in the model: the initial wave velocity, $u_0$, and the
velocity correlation length, $L$.
The three-minute periods were found to be freely propagating
(i.e.\ no driving force present after the initial excitation)
dispersive waves
whose frequency rapidly dropped due to energy loss. The damping of
high-frequency waves first, follwed by lower frequency waves later
explained the drop in power beyond the cutoff frequency that was
initially unexplained. The source of energy powering these
oscillations in the first place was attributed to either the continuous
photospheric motions caused by convective turbulence
in the solar interior, or velocity impulses from impulsive disturbances
in the upper photosphere.

%\bibliography{reffile}
\end{document}
