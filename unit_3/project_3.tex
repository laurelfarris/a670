\documentclass[12pt]{article}
\usepackage[margin=1in]{geometry}
\setlength{\parindent}{0em}
\setlength{\parskip}{1em}
\usepackage{fancyhdr}
\pagestyle{fancy}
\pagenumbering{gobble}
\rhead{?? April 2016}
\chead{ASTR 670 --- Project 3 (paper summary)}
\lhead{Laurel Farris}

%\usepackage[super]{natbib}
%\usepackage[superscript,biblabel]{cite}
\usepackage{natbib}
\bibliographystyle{plain}
\usepackage{hyperref}

\begin{document}
\begin{centering}
    {\large ``ACOUSTIC WAVES GENERATED BY IMPULSIVE DISTURBANCES IN A
    GRAVITATIONALLY STRATIFIED MEDIUM''}
\end{centering}
\begin{centering}
    -Jongchul Chae \& Philip R. Goode
\end{centering}

% Main topic
Disturbances observed in the atmosphere of the sun are interpreted as different
kinds of waves, depending on properties such as their speed, lifetime,
oscillation period, etc. How these disturbances are generated in the first
place has been somewhat of a mystery until recently.

% Data/Methods
% Results

%\bibliography{reffile}
\end{document}
